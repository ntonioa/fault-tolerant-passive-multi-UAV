\section{Introduction}
\label{sec:introduction}
In recent years, the landscape of transportation has witnessed a transformative shift towards electromobility, reflecting a global commitment to sustainable and environmentally conscious practices. This evolution is propelled by advancements in electric vehicle technology, coupled with an increasing societal awareness of the urgent need to reduce carbon emissions in the transport sector. As the number of electric vehicles increases, the demand for efficient charging strategies becomes paramount to ensure seamless integration into the existing infrastructure.

Consider the following scenario: a fleet of PEVs are connected to a charging station. The charging stations are linked to a central controller, which is responsible for managing the charging process. The controller has to decide the charging rate of each vehicle, taking into account their charging requirements and the power limits of the grid connection. The goal is to minimize the total charging cost tracking the aggregated power reference (i.e. desired value for the total power withdrawn from the grid), while ensuring that the charging process is completed according to the preferences of each driver, expressed in terms of final state of charge and charging time.

Typically, the problem is addressed with Model Predictive Control (MPC) techniques, which are well suited for the management of complex systems with multiple constraints. In such instances, the decentralized approach involves dividing the problem into multiple subproblems, which are collaboratively solved by both the plug-in electric vehicles and the control center. This contrasts with the centralized method where data is collected from PEVs, and a singular computation is performed at the control center, which may become impractical for large-scale systems due to the escalating computational burden associated with an increasing number of variables.

Decentralized approaches have been proposed in the past literature. 
Vujanic et al. (in their work \autocite{VUJANIC2016144}) were among the pioneering researchers exploring scalable PEV charging control solutions by employing decentralized approaches to solve mixed-integer linear problems. Further developments have been proposed by Falsone et al. (in their article \autocite{FALSONE2019141}, for which significant improvements were developed in the recent work \autocite{MANIERI20235919} by Manieri et al.).

The work presented in this document is a thorough review of the developements proposed by Liberati et al. in their article \autocite{10194623}. Their main contributions lie in modeling the charging process of each PEV as a semi-continuous variable and in the inclusion of the additional objective of tracking a desired aggregated power profile. Here, adopting the solution advanced by Falsone et al. in \autocite{FALSONE2019141}, the authors highlight the differences between a centralized and a decentralized approach, both in terms of problem formulation and solution algorithm. The two approaches are finally compared in terms of computational complexity and performance, by means of simulations.

Table \ref{tab:nomenclature} at the end of the document summarizes the nomenclature used.