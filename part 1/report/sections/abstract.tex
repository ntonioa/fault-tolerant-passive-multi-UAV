\begin{abstract}
    This term paper presents a decentralized approach to the problem of charging a fleet of plug-in electric vehicles (PEVs) in a smart grid, where global and local constraints and objectives are considered together with the model of the power grid by means of a MPC strategy. The representation of the charging problem involves a combination of continuous and integer variables (specifically of Boolean nature). It is formulated as a mixed-integer linear programming (MILP) problem, which is solved using a subgradient method. The proposed approach is compared with a centralized one, which is typically solved by a branch-and-bound algorithm. The results show that the decentralized approach is able to achieve a performance close to the centralized one, while reducing the computational burden, especially for large fleets of vehicles. These findings are supported by means of simulations performed on a test case.
\end{abstract}