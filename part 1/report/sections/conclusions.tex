\section{Conclusions}
In the present document, two different approaches to the PEV charging problem have been presented. The first one is a centralized approach, where the problem is solved by a single entity that has access to all the information about the system. The second one is a decentralized approach, where the problem is solved by a set of agents that have access only to local information, supervised by a central entity. The analysis of the two approaches has been carried out both in qualitative (level of performance) and quantitative (computational burden) terms.

The centralized algorithm manages to find a solution that satisfies the requests of the PEV owners, while keeping the power flow in the network within the limits and almost perfectly equal to the reference. However, it is important to highlight that its computational load is very high, and is not suitable for real-time applications. In fact, the computational time is of the order of minutes, and it is not possible to reduce it significantly without losing the optimality of the solution. Moreover, this controller requires a lot of sensitive information about the PEVs, such as the arrival and departure times, the battery capacity and the required energy. 

On the other hand, the decentralized algorithm identifies a feasible solution that is sufficiently close to the centralized one in a noticeably shorter time. Besides, the central unit does not require any sensitive information about the PEVs, as all the necessary data is coded in the form of the local tentative power schedules.