\subsection{Improved computation of the tightening array}
Mainly relying on the work of Vujanic et al., Falsone et al. (2019)\supercite{falsone} proposes a decentralized iterative procedure allowing to compute in finite time a less conservative and well performing solution for \ref{eq:MILP}.\\
As a matter of fact, in the previous part, the larger the value of $\rho$, the more the feasibility of the solution is hampered because of the reduction of the maximum available resources after the restriction. In order to recover feasibility of part of the solutions, the amount of resource tigthening (through $\rho$) is iteratively updated based on the candidate solutions explored up to that moment. This guarantees that the value of $\rho$ is always lower or equal to the tightening array proposed in \ref{eq:rho_vuj}, which represents the worst-case scenario. Consequences include the possibility of applying the procedure to a larger class of problems with lower computational effort due to the finite-time convergence of the sub-gradient algorithm.\\
The main contribution provided in this work is summed up in the following law:
\begin{align*}
    \rho(k) = \beta \cdot \max_{p} \left\{ \max_{\tau \leq k} A_p v_p(\tau) - \min_{\tau \leq k} A_p v_p(\tau) \right\},
\end{align*}
which constitutes a monotonically non-decreasing sequence whose computation at instant $k$ is based on the solutions $v_p(\tau)$ for $\tau \leq k$ of the inner problem \ref{eq:inner}, with $\lambda = \lambda(k)$, which depends on the restricted resource array $\bar{b}$ through the last available value of $\rho$, i.e., $\rho'(k-1)$. \\ Vujanic et al.'s approach would need, in principle, to solve $\beta$ MILPs for an infinite number of iterations, unlike Falsone et al.'s one, that only requires to solve $\beta$ MILPs for a finite number of times.

\subsubsection{Application example}
In the final pages of Falsone et al. (2019)\supercite{falsone}, the authors refer directly to the PEV recharging problem presented in Vujanic et al. (2014)\supercite{vujanic} to make a comparison between the performance of the two decentralized approaches.\\
The first aspect where improvements are observed is the reduction in conservativity: the infinity-norm of the resource restriction array $\rho$ of Falsone et al. is equal to $50\%$ of that Vujanic et al.'s, regardless of the number of vehicles considered. Consequently, the authors report having simulated scenarios where, given stricter global constraints on the maximum aggregate power, a feasible solution was found only by using Falsone et al.'s procedure.\\
Moreover, in both solutions the optimality gap keeps decreasing with the number of PEVs according to \ref{eq:ratio}, but the convergence is faster in Falsone et al.'s approach.\\