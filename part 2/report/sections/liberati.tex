\subsection{Semi-continouos decision variables and reference tracking}
In Liberati et al. (2023)\supercite{liberati}, the authors worked towards two goals, both introduced for the purpose of applying the previously introduced framework to a realistic PEV charging setting. The first one is to model the decision variables $v_p$ possibly as semi-continuous variables (i.e., they are allowed to be either null or continuous); the second is to combine the coupling constraint with the goal of tracking a global reference.

The first objective is achieved with the introduction of auxiliary boolean variables $\delta_p \in \{0,1\}$ that only influence the local domains $V_p$.\\
On the other hand, in order to achieve the second goal, in principle one has to minimize (component-wise) the distance between a linear functional of the decision variables $\mathcal{F} (v_1, \dots, v_m)$ and the chosen reference $f^\star$. In light of this, the renewed cost function would be:
\begin{align*}
    \min_{v_1, \dots, v_m} \left\{ \mathbbm{1}^T |\mathcal{F} (v_p) - f^\star| + \sum_{p = 1}^m c_p^T v_p\right\},
\end{align*}
where $\mathbbm{1}$ is a vector of ones of appropriate dimension.\\
The problem is that the last expression is nonlinear, but it can be linearized by introducing an auxiliary decision variable $t$ that satisfies the following inequality constraints:
\begin{align*}
    -t \leq \mathcal{F} (v_1, \dots, v_m) - f^\star \leq t.
\end{align*}
The final formulation of the primal problem is:
\begin{align*}
    \min_{v_1, \dots, v_m, t} & \left\{ \mathbbm{1}^T t + \sum_{p = 1}^m c_p^T v_p\right\} \\
    \text{subject to: } & \sum_{p=1}^{m} A_p v_p \leq b, \tag{$\mathcal{P}$} \label{eq:MILP_liberati} \\ 
    & -t \leq \mathcal{F} (v_1, \dots, v_m) - f^\star \leq t, \quad \\
    & v_p \in V_p, \quad p = 1, \dots, m.
\end{align*}

\subsubsection{Application example}
Within the PEV charging problem (already mentioned in both Vujanic et al. (2014)\supercite{vujanic} and Falsone et al. (2019)\supercite{falsone}), the innovations introduced by Liberati et al. (2023)\supercite{liberati} translate into modulating the charging and discharging power of each vehicle as a semi-continuous variable, modeling the realistic case for which the power can be either zero or take on values within an allowed range, and adding a desired aggregated power profile to be tracked globally.\\
The results show that the proposed approach is able to track the reference profile with a good approximation, while still respecting the constraints on the maximum aggregated power. 