\section{Conclusions}
In this paper, we presented the state of the art of the decentralized approach to solve large-scale mixed-integer linear programming (MILP) problems, with a specific focus on the application to the control of plug-in electric vehicles (PEVs) charging in a smart grid environment. 

Through the studies provided in Vujanic et al. (2014)\supercite{vujanic}, we have seen how the problem of PEV charging can be formulated as a MILP problem, and how a multi-agent decentralized approach can be used to find a solution satisfying global coupling constraints through the use of a resource restriction array. Vujanic et al.'s method has been further improved in Falsone et al. (2019)\supercite{falsone}, which provides a less-conservative and better-performing iterative method to compute the restriction array, and therefore to find a solution. Furthermore, we have seen the strategy of Liberati et al. (2023)\supercite{liberati} to exploit the structure of a MILP problem and suggest a realistic modeling of the PEV recharge problem (introducing global reference tracking and semi-continuous variables). 

Finally, the main focus of this paper has been on the work of Manieri et al. (2023)\supercite{manieri}. Here, the authors propose a novel minimally-conservative method to find a solution for the MILP problem, once again based on the idea of resource restriction, with the additional perk of allowing the value of the restriction array both to increase and to decrease. Although the procedure in issue promises great performance in theory, we have found that gaining an advantage from an application point of view is somewhat complicated and requires appropriate trade-offs. Certainly, in order to obtain even better results in this direction, there is a need for further investigation, especially in the case where the coupling constraints have large size, or when the number of vehicles is scarce.
